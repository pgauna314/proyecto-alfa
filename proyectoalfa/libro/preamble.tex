% Archivo: preamble.tex
% Configuración básica del documento
\documentclass[12pt, a4paper]{book} % Clase book
\usepackage[spanish]{babel} % Idioma español
\usepackage[utf8]{inputenc} % Codificación UTF-8
\usepackage[T1]{fontenc} % Mejora la compatibilidad de fuentes

% Paquetes matemáticos
\usepackage{amsmath, amssymb, amsthm} % Para ecuaciones y símbolos matemáticos
\usepackage{siunitx} % Para unidades (ej. \SI{100}{\kilo\watt})
\usepackage{mathtools} % Mejoras para ecuaciones

% Gráficos y figuras
\usepackage{graphicx} % Para incluir imágenes
\usepackage{wrapfig} % Para envolver texto alrededor de figuras
\usepackage{subcaption} % Para subfiguras
\usepackage{tikz} % Para gráficos vectoriales
\usetikzlibrary{shapes, arrows.meta, positioning} % Librerías de TikZ
\usepackage{pgfplots}
\pgfplotsset{compat=1.18}
\usepackage{soul}
\usepackage[most]{tcolorbox} % Esto ya incluye skins

\newtcolorbox{definicion}[1]{%
    enhanced,
    colback=blue!5,
    colframe=blue!60!black,
    boxrule=0.5pt,
    leftrule=4pt,
    sharp corners,
    fonttitle=\bfseries,
    title={#1}
}
% Tablas
\usepackage{array} % Mejoras para tablas
\usepackage{booktabs} % Tablas profesionales
\usepackage{longtable} % Tablas largas

% Diseño y formato
\usepackage{fancyhdr} % Encabezados y pies de página personalizados
\usepackage{titlesec} % Personalización de títulos de sección
\usepackage{setspace} % Espaciado entre líneas
\usepackage{geometry} % Configuración de márgenes
\geometry{a4paper, margin=1in} % Márgenes de 1 pulgada


% Paquetes necesarios para mejorar las tablas
\usepackage{siunitx} % Para formatear números con separadores de miles
\usepackage{booktabs} % Para mejorar el diseño de las tablas
\usepackage{array} % Para formatear columnas
\usepackage{tabularx} % Para ajustar el ancho de las tablas
\usepackage{ragged2e} % Para justificar texto en columnas

% Definición de una columna centrada con ajuste automático
\newcolumntype{C}{>{\Centering\arraybackslash}X} % Columna centrada en tabularx



% Hipervínculos y referencias
\usepackage{hyperref} % Para enlaces y referencias
\hypersetup{
    colorlinks=true,
    linkcolor=blue,
    urlcolor=blue,
    citecolor=blue,
    pdftitle={Termodinámica de los Procesos Productivos},
    pdfauthor={Tu Nombre},
    pdfsubject={Termodinámica para ingenieros de proceso},
    pdfkeywords={Termodinámica, Procesos Productivos, Ingeniería Química}
}

%-------------------------------------------------------------------------
\usepackage[backend=biber,giveninits=true,style=authoryear,maxcitenames=2,maxbibnames=9,natbib=true,url=false,date=year]{biblatex} % Use the bibtex backend with the authoryear citation style (which resembles APA)


%AGREGADO para bibliografia por capitulos

\defbibheading{subchap}[\referencias3]{%
    \section*{#1}%
    \addcontentsline{toc}{section}{#1}%
    \markboth{\MakeUppercase{#1}}{\MakeUppercase{#1}}}


\addbibresource{referencias.bib} % The filename of the bibliography
%-------------------------------------------------------------------------

% Bibliografía
%\usepackage[backend=biber, style=apa]{biblatex} % Gestión de bibliografía
%\addbibresource{referencias.bib} % Archivo de referencias

% Misceláneos
\usepackage{enumitem} % Personalización de listas
\usepackage{csquotes} % Mejora las comillas
\usepackage{lipsum} % Para texto de relleno (opcional, eliminar en producción)
\usepackage[version=4]{mhchem} %formulas quimicas, reacciones, figuras moleculares
\usepackage[xcolor]{mdframed} %recuadros para definiciones
\usepackage{soul} %para highlight texto

% Configuración adicional
\setlength{\parindent}{0pt} % Sin sangría en los párrafos
\setlength{\parskip}{1em} % Espaciado entre párrafos
\renewcommand{\baselinestretch}{1.2} % Espaciado entre líneas
\renewcommand{\tablename}{Tabla}
\addto\captionsspanish{\renewcommand{\tablename}{Tabla}}