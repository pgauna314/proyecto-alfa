\chapter*{Prólogo}
\addcontentsline{toc}{chapter}{Prólogo}

\chapter*{Prólogo}
\addcontentsline{toc}{chapter}{Prólogo}

Si bien en ingeniería la Termodinámica no es una herramienta específica de diseño, abarca los conceptos indispensables para comprender, optimizar y mejorar cualquier sistema energético o industrial. Desde la generación de potencia hasta el diseño de procesos químicos, refrigeración, climatización y almacenamiento de energía, la Termodinámica actúa como el lenguaje universal que conecta las leyes fundamentales de la naturaleza con aplicaciones prácticas en el mundo real.

En un mundo globalizado, donde las soluciones tecnológicas y energéticas suelen diseñarse bajo estándares internacionales, es fácil perder de vista la importancia de adaptar estos conocimientos a las realidades locales. Este libro nace con la convicción de que la Termodinámica no es solo \textit{la rama de la Física que estudia la energía y sus transformaciones}, sino también una herramienta poderosa para abordar desafíos concretos en nuestro entorno inmediato. Más allá de ser una ciencia abstracta, la Termodinámica nos permite interpretar fenómenos cotidianos y complejos, desde el funcionamiento de una simple cafetera hasta el diseño de centrales eléctricas que alimentan ciudades enteras.

El enfoque local escogido para este texto es más que una estrategia pedagógica: es una declaración de principios. Al centrar este libro en ejemplos, casos de estudio y aplicaciones relevantes para nuestra región, buscamos resaltar cómo los ingenieros, técnicos y científicos pueden convertirse en \textit{agentes críticos y transformadores} de su entorno. Aquí no se trata de aplicar fórmulas aprendidas de memoria, sino de \textit{leer el mundo energético} para luego transformarlo. Esta idea —central en la pedagogía de Paulo Freire— guía nuestro propósito: la educación no debe depositar conocimientos en mentes pasivas, sino despertar la capacidad de cuestionar, crear y actuar con autonomía.

La soberanía energética, por ejemplo, no es solo un tema político, sino también técnico. Depender de tecnologías importadas o de combustibles externos limita nuestra capacidad de decisión como Nación. Por ello, este libro invita a reflexionar: ¿cómo podemos diseñar, optimizar y gestionar sistemas que aprovechen los recursos disponibles localmente, promoviendo así una mayor independencia energética y económica? La respuesta no está en manuales extranjeros, sino en la articulación entre conocimiento científico riguroso y saberes arraigados en nuestra tierra.

Además, la Termodinámica ofrece un marco conceptual que permite evaluar la eficiencia y sostenibilidad de los sistemas existentes, identificando oportunidades para innovar y reducir dependencias externas. En este sentido, el uso de fuentes renovables como la energía solar, eólica o hidráulica no solo se justifica por su menor impacto ambiental, sino también por su potencial para democratizar el acceso a la energía, especialmente en comunidades rurales o remotas donde las infraestructuras tradicionales son insuficientes o inexistentes. Así, la técnica se vuelve política: no se trata solo de eficiencia, sino de justicia energética.

La profesión de Ingeniero de Procesos —con la Termodinámica como una de sus herramientas fundamentales— tiene un rol crucial en la generación de empleo y el fortalecimiento de las economías locales. Las centrales térmicas, hidroeléctricas y nucleares, así como las plantas de tratamiento de aire y los sistemas frigoríficos, son infraestructuras que requieren mano de obra calificada, pensamiento crítico y compromiso con el entorno. Al formar profesionales capaces de entender y mejorar estos sistemas desde una perspectiva local y crítica, no solo creamos oportunidades laborales, sino que fomentamos una cultura de innovación, autosuficiencia y dignidad técnica.

Finalmente, este libro no pretende ser un mero compendio de ecuaciones y gráficos. Más bien, busca inspirar a los lectores a asumir su rol como sujetos activos del cambio. La ciencia, cuando se ejerce con conciencia social y ambiental, se convierte en una fuerza poderosa para el bien común. Desde la reducción de emisiones contaminantes hasta la implementación de tecnologías limpias y renovables, cada paso hacia una mayor eficiencia energética es, también, un paso hacia un mundo más justo y equitativo.

Este texto es, en suma, una invitación a mirar más allá de las fórmulas: a ver en ellas no solo leyes físicas, sino posibilidades de construcción colectiva. Arraigado en nuestras raíces locales, pero con la mirada puesta en el horizonte global, este libro aspira a formar no solo ingenieros competentes, sino ciudadanos conscientes, capaces de transformar la realidad desde la ética del trabajo técnico y la solidaridad social.