\documentclass[12pt,a4paper]{article}
\usepackage[utf8]{inputenc}
\usepackage[spanish]{babel}
\usepackage{booktabs}
\usepackage{geometry}
\usepackage{longtable}
\usepackage{amsmath}

\geometry{margin=1in}

\begin{document}

\section*{Sección A: Observatorio de la Matriz Energética Argentina}
Este inventario consolida las principales centrales del país para su modelización en el \textit{Tomo I}. La generación se expresa en Potencia Instalada (MW), siendo el aporte real dependiente del despacho de CAMMESA.

\subsection*{1. Centrales Térmicas de Ciclo Combinado (Mayor Eficiencia)}
Representan el eje del estudio por su complejidad termodinámica (Ciclos Brayton + Rankine).

\begin{longtable}{p{4cm} p{3.5cm} p{2.5cm} p{3cm}}
\toprule
\textbf{Central} & \textbf{Ubicación} & \textbf{Potencia (MW)} & \textbf{Referencia Termo} \\
\midrule
C.T. Genelba & Marcos Paz, BA & 1253 & Alta presión / Litoral \\
C.T. San Miguel de Tucumán & Tucumán & 830 & Clima Subtropical \\
C.T. Manuel Belgrano & Campana, BA & 873 & Polo Industrial Delta \\
C.T. José de San Martín & Timbúes, SF & 875 & Ribera Río Paraná \\
C.T. Loma de la Lata & Neuquén & 640 & Boca de Pozo \\
C.T. Güemes & Salta & 361 & Efecto de Altitud \\
C.T. Pilar & Córdoba & 466 & Sumidero de aire \\
\bottomrule
\end{longtable}

\subsection*{2. Centrales Térmicas de Vapor (Ciclos Rankine Convencionales)}
Fundamentales para explicar procesos de recalentamiento y regeneración.

\begin{longtable}{p{4cm} p{3.5cm} p{2.5cm} p{3cm}}
\toprule
\textbf{Central} & \textbf{Ubicación} & \textbf{Potencia (MW)} & \textbf{Referencia Termo} \\
\midrule
C.T. Costanera & CABA & 2324 & Gran Escala \\
C.T. Puerto Nuevo & CABA & 589 & Histórica / Diseño \\
C.T. San Nicolás & San Nicolás, BA & 650 & Multicombustible \\
C.T. Piedrabuena & Bahía Blanca & 620 & Condensación Marina \\
C.T. Río Turbio & Santa Cruz & 240 & Ciclo Carbón / NOA \\
\bottomrule
\end{longtable}

\subsection*{3. Centrales Nucleares (Vapor Saturado)}
Eje de la innovación pedagógica en el uso de diagramas de fases y títulos de vapor.

\begin{longtable}{p{4cm} p{3.5cm} p{2.5cm} p{3cm}}
\toprule
\textbf{Central} & \textbf{Ubicación} & \textbf{Potencia (MW)} & \textbf{Referencia Termo} \\
\midrule
Atucha I & Lima, BA & 362 & PHWR / Histórica \\
Atucha II & Lima, BA & 745 & PHWR / Gran Escala \\
Embalse & Córdoba & 648 & CANDU / Lago \\
\bottomrule
\end{longtable}

\subsection*{Resumen de Aporte por Tecnología}
Para el análisis de soberanía energética y publicaciones técnicas, se considera el siguiente aporte porcentual aproximado a la demanda neta nacional:
\begin{itemize}
    \item \textbf{Térmica Fósil (Gas/Fuel):} $\approx 60-65\%$ (Eje del Tomo I).
    \item \textbf{Hidroeléctrica:} $\approx 20-25\%$ (Eje del Tomo II: Procesos Hidráulicos).
    \item \textbf{Nuclear:} $\approx 4-7\%$ (Anexo Especial).
    \item \textbf{Renovables (Eólica/Solar):} $\approx 10-15\%$ (Análisis de Complementariedad).
\end{itemize}

\section*{Sección A: Observatorio de la Matriz Energética (Continuación)}

\subsection*{4. Metodología de Análisis del Observatorio}
Para cada central del inventario, el ecosistema PIER aplica un protocolo de ingeniería inversa para transformar datos públicos de despacho en modelos termodinámicos de alta fidelidad.

\subsubsection*{A. Determinación de Estados mediante la App}
El software desarrollado para este proyecto utiliza las tablas de vapor del \textit{Anexo B} del libro para fijar los estados termodinámicos. Se asumen los siguientes rendimientos isoentrópicos estándar para el parque nacional:
\begin{itemize}
    \item \textbf{Turbinas de Vapor (TV):} $\eta_{iso} \approx 0.85 - 0.90$.
    \item \textbf{Turbinas de Gas (TG):} $\eta_{iso} \approx 0.82 - 0.88$.
    \item \textbf{Bombas de Alimentación:} $\eta_{iso} \approx 0.75$.
\end{itemize}

\subsubsection*{B. Modelado de Condiciones Ambientales Regionales}
A diferencia de los textos tradicionales, el Observatorio ajusta el sumidero térmico según la ubicación:
\begin{itemize}
    \item \textbf{Nodos Litoral/Ribera:} Condensación por agua de río ($T_{amb} \approx 15-25^\circ C$).
    \item \textbf{Nodos NOA/Cuyo:} Condensación por torres de enfriamiento o aire ($T_{amb} \approx 25-35^\circ C$).
    \item \textbf{Efecto Altitud:} Ajuste de la presión de admisión de aire ($P_{atm}$) según cota msnm.
\end{itemize}

\subsection*{5. Matriz de Variables de Diseño para Simulación}
Esta tabla define los "inputs" que los alumnos y becarios utilizarán en la App para replicar el comportamiento de las centrales:

\begin{longtable}{p{3.5cm} p{2.5cm} p{2.5cm} p{2.5cm} p{3cm}}
\toprule
\textbf{Tecnología} & \textbf{$P_{caldera}$} & \textbf{$T_{sobrecal.}$} & \textbf{$P_{condens.}$} & \textbf{Eficiencia Ref.} \\
\midrule
Ciclo Combinado & 80-110 bar & 520-540 °C & 0.08-0.12 bar & 52\% - 58\% \\
Vapor Convenc.  & 120-160 bar & 535-545 °C & 0.05-0.10 bar & 35\% - 40\% \\
Nuclear (PHWR)  & 45-55 bar  & 260-275 °C & 0.05 bar & 30\% - 32\% \\
\bottomrule
\end{longtable}

\section*{6. Impacto en la Producción Científica}
Los datos procesados en esta sección alimentarán las siguientes publicaciones previstas:
\begin{enumerate}
    \item \textit{``Análisis comparativo de la eficiencia exergética en el SADI: Impacto de la ubicación geográfica''}.
    \item \textit{``Optimización de ciclos combinados bajo condiciones climáticas extremas en Argentina''}.
\end{enumerate}
\end{document}