\documentclass[12pt,a4paper]{article}
\usepackage[utf8]{inputenc}
\usepackage[spanish]{babel}
\usepackage{geometry}
\usepackage{titlesec}
\usepackage{enumitem}

\geometry{margin=1in}
\titleformat{\section}{\large\bfseries}{\thesection}{1em}{}[\titlerule]

\title{Manual de Diseño y Requerimientos: Toolbox ``Termo-Procesos'' \\ \large Integración Libro-Software para UNGS/UTN Delta}
\author{Dr. Pablo S. Gauna}
\date{Versión 1.0 - 2025}

\begin{document}

\maketitle

\section{Objetivo de la Aplicación}
La App tiene como propósito transformar la consulta estática de tablas termodinámicas en una experiencia visual e interactiva. Debe permitir al alumno modelar el parque generador argentino (Sección A) y visualizar los estados en diagramas de fases en tiempo real.

\section{Módulos Funcionales}

\subsection{Módulo 1: Visualizador de Estados Dinámico}
Este es el núcleo de la innovación. En lugar de una simple calculadora:
\begin{itemize}
    \item \textbf{Input:} El usuario ingresa dos propiedades independientes ($P, T$, $P, x$, etc.).
    \item \textbf{Gráfico Dinámico:} La App dibuja automáticamente la campana de saturación del agua y ubica el punto de estado.
    \item \textbf{Innovación en Tablas:} El usuario puede ``deslizar'' un punto sobre una isóbara y ver cómo cambian los valores de entalpía y entropía sin salir del gráfico.
\end{itemize}

\subsection{Módulo 2: Biblioteca de Centrales Argentinas}
\begin{itemize}
    \item \textbf{Base de Datos integrada:} Conexión directa con las 5+ centrales del Observatorio.
    \item \textbf{Carga de Casos:} Al seleccionar ``C.T. Güemes'', la App carga automáticamente las presiones de alta y baja, y el tipo de ciclo (Combinado).
    \item \textbf{Análisis de Sensibilidad:} Capacidad de modificar la temperatura del sumidero térmico (río vs. torre) para observar el impacto en la eficiencia del ciclo.
\end{itemize}

\section{Interfaz de Usuario (UI/UX)}
\begin{enumerate}
    \item \textbf{Panel Izquierdo (Inputs):} Selección de sustancia, selección de central del observatorio y parámetros de entrada.
    \item \textbf{Panel Central (Visualización):} Gráfico $T-s$ o $P-h$ en alta resolución (formato vectorial para informes).
    \item \textbf{Panel Derecho (Resultados):} Cuadro de propiedades (h, s, u, v) y cálculo de rendimientos ($\eta_{th}, \eta_{ex}$).
\end{enumerate}

\section{Integración Editorial y Académica}
\begin{itemize}
    \item \textbf{Exportación LaTeX:} La App debe generar el código de la tabla de resultados para que el alumno lo pegue en su trabajo práctico.
    \item \textbf{Sincronización con el Tomo I:} Los ejercicios del libro tendrán un código de acceso o QR que la App reconocerá para cargar el problema automáticamente.
\end{itemize}

\end{document}