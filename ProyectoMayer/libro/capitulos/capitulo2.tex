% Archivo: capitulos/capitulo2.tex
\chapter{Conceptos Fundamentales}

\section{Objetivos de este capítulo}
En este capítulo, abordaremos los conceptos fundamentales de \textbf{sistema}, \textbf{balance de materia} y \textbf{balance de energía}, aplicándolos al funcionamiento de una central térmica. En lugar de presentar definiciones abstractas desde el inicio, analizaremos cómo estos principios se aplican en la generación de electricidad, para finalmente formalizar los conceptos clave.



\section{Introducción: Central Térmica}
En capítulos previos, mencionamos que una central térmica es aquella que impulsa la turbina con vapor de agua, el cual se produce gracias al calor generado por la combustión de un combustible fósil. Veamos entonces las etapas básicas de una central térmica, como se muestra en la Figura \ref{fig:CentralTermica}:

\begin{itemize}
    \item \textbf{Vaporización del agua}: Una corriente de agua se vaporiza gracias al calor producido por la combustión de un combustible fósil. El agua circula por un conjunto de cañerías que atraviesan una caldera, donde ocurre la combustión.
    \item \textbf{Expansión del vapor}: El vapor generado (a alta temperatura y presión) se dirige mediante cañerías a la turbina. Allí, el vapor mueve los álabes de la turbina, transformando la energía acumulada en el fluido en energía mecánica.
    \item \textbf{Condensación del vapor}: El vapor que abandona la turbina (a baja presión y temperatura) se dirige a un intercambiador de calor, donde se condensa por completo al retirarle parte de su energía.
    \item \textbf{Bombeo del agua}: El agua en estado líquido se bombea a alta presión para reingresar a la caldera, continuando así el ciclo de manera cíclica.
\end{itemize}

\begin{figure}[htbp]
    \centering
    \includegraphics[scale=0.75]{graphics/centraltermica.jpg}
    \caption{Procesos básicos involucrados en una central térmica.}
    \label{fig:CentralTermica}
\end{figure}



\section{Sistema abierto o volumen de control}
Teniendo en mente el funcionamiento de una central térmica, supongamos que queremos analizar termodinámicamente el agua que circula por una cañería dentro de la caldera. Para ello, tracemos una línea imaginaria que abarque esa cañería, como se muestra en la Figura \ref{fig:EleccionSistema} (recuadro rojo). Esta elección nos permite definir tres regiones:

\begin{figure}[htbp]
    \centering
    \includegraphics[scale=0.75]{graphics/SistemaAbiertoCaldera.jpg}
    \caption{Elección de un sistema.}
    \label{fig:EleccionSistema}
\end{figure}

\begin{itemize}
    \item \textbf{Sistema}: La región dentro del recuadro, que es el objeto de nuestro análisis.
    \item \textbf{Medio o entorno}: La región fuera del recuadro, que interactúa con el sistema.
    \item \textbf{Límite o frontera}: La línea que delimita el recuadro, que separa el sistema del medio.
\end{itemize}

Si observamos el recuadro rojo (el límite del sistema), notaremos que hay una corriente de agua que entra y sale de él, lo que significa que la región es atravesada por una cantidad determinada de materia. A este tipo de sistema se lo conoce como \textbf{sistema abierto} o \textbf{volumen de control}. Además, dado que el agua se vaporiza gracias al calor de la combustión, el sistema también intercambia energía con el medio.

Podemos entonces establecer la siguiente definición:

\begin{mdframed}[backgroundcolor=gray!20, linecolor=black, linewidth=1.0pt]
\centering
Un \textbf{sistema abierto} (o volumen de control) es una región del espacio seleccionada para su análisis, que intercambia materia y energía con el medio que lo rodea.
\end{mdframed}


\subsection{Lo que tenés que saber sobre los sistemas abiertos}
Las definiciones académicas son importantes, pero lo esencial es aprender a aplicar estos conceptos en la resolución de problemas. Para ello, tené en cuenta lo siguiente:

\begin{enumerate}
    \item Elegir un sistema correctamente permite plantear ecuaciones que facilitan la resolución de problemas.
    \item La selección del sistema es clave en el análisis termodinámico; desarrollar criterio para elegirlo es parte fundamental del aprendizaje.
    \item Inicialmente, este criterio se adquiere mediante prueba y error.
    \item La elección del sistema no tiene que coincidir con algo físico: en este caso, elegimos la cañería dentro de la caldera, pero podríamos haber seleccionado toda la caldera, la turbina, el condensador, la bomba o toda la central.
\end{enumerate}

\hl{Un último detalle: además de los sistemas abiertos, hay otros dos tipos, sobre los que hablaremos más adelante: el sistema cerrado y el sistema aislado.}



\section{Balance de materia}
El balance de materia es una herramienta fundamental en el análisis de sistemas termodinámicos. Siguiendo el principio de conservación de la masa, esta no se crea ni se destruye dentro de un sistema (salvo en procesos nucleares o atómicos, que no aplican en nuestro caso). Esto nos lleva a la ecuación general del balance de materia:

\begin{equation}
\text{acumulación} = \text{entrada} - \text{salida} + \text{generación} - \text{consumo}
\end{equation}

En la mayoría de los sistemas de ingeniería química sin reacción química, los términos de generación y consumo se anulan, simplificando la ecuación a:

\begin{equation}
\label{ec:BalanceMateriaEstacionario}
\sum_e \dot{m}_e = \sum_s \dot{m}_s
\end{equation}

Esto indica que la cantidad de masa que entra al sistema es igual a la que sale, siempre que estemos en estado estacionario. En una central térmica, este principio se cumple en cada etapa del ciclo. Por ejemplo, el flujo de agua que ingresa a la caldera en estado líquido debe ser igual al flujo de vapor que sale, salvo pequeñas pérdidas despreciables.

Dado que el agua en sus diferentes estados \textbf{circula} constantemente dentro del ciclo térmico, es más útil analizar no una cantidad fija de masa, sino el flujo de masa que atraviesa el sistema en función del tiempo. Para ello, utilizamos el \textbf{caudal másico}, denotado como $\dot{m}$. La notación con un punto sobre la \( m \) indica que se trata de una magnitud expresada por unidad de tiempo, es decir, un flujo másico en \(\text{kg/s}\).



\section{Balances de energía}
\subsection{El Primer Principio}
No tengo ninguna duda de que, palabras más, palabras menos, aprendiste en asignaturas previas (o incluso en la escuela secundaria) que la energía no se crea ni se destruye, sino que se transforma. En otras palabras, ya tenés bien claro que la energía se conserva. Lo que quizás no tengas tan claro es que esto no es ni más ni menos que el \textbf{Primer Principio de la Termodinámica}, y lo emplearemos como uno de los instrumentos de análisis. Antes de seguir, algunas aclaraciones que considero necesarias sobre principios, leyes y teorías.

\subsubsection{Sobre leyes, principios y teorías}
En el contexto de la ciencia, una ley (o principio) \textbf{describe cierto fenómeno natural por medio de la relación de algunas variables}. Una ley no se demuestra, porque su función es simplemente describir un comportamiento observado en la naturaleza. Sin embargo, una teoría es un conjunto de ideas o suposiciones comprobadas que \textbf{explican} por qué ocurre ese fenómeno.

Para entender mejor esta diferencia, veamos un ejemplo cotidiano:

\begin{itemize}
    \item \textbf{Ley}: Imagina que observas que cada vez que sueltas un objeto, este cae al suelo. Podrías formular una ley que diga: "Todos los objetos cerca de la superficie de la Tierra caen con una aceleración constante de \(9.8 \, \text{m/s}^2\)". Esta ley describe lo que ocurre, pero no explica por qué sucede.
    \item \textbf{Teoría}: Ahora, para explicar por qué los objetos caen, necesitas una teoría. La Teoría de la Gravitación de Einstein (Relatividad General) explica que los objetos caen porque la masa de la Tierra curva el espacio-tiempo a su alrededor, y los objetos simplemente siguen esa curvatura. Esta teoría no se deduce de la ley, sino que se construye a partir de observaciones, experimentos y datos que validan su explicación.
\end{itemize}

En este ejemplo, la ley describe el comportamiento (los objetos caen), mientras que la teoría explica la causa (la curvatura del espacio-tiempo). Es importante destacar que una teoría no se convierte en ley, ni una ley se deduce de una teoría. Ambas son herramientas complementarias: la ley describe, y la teoría explica.

Volviendo al caso de Newton: él formuló la Ley de Gravitación Universal, que describe cómo dos cuerpos se atraen en función de sus masas y la distancia que los separa. Sin embargo, la explicación de por qué ocurre esta atracción no estaba en su ley, sino que llegó más tarde con la Teoría General de la Relatividad de Einstein.

Retomando con nuestra querida Termodinámica: está estructurada sobre cuatro leyes o principios que describen distintos comportamientos de la energía, los cuales iremos viendo conforme avancemos en este libro.


\subsection{La ecuación del Primer Principio para sistemas abiertos}
En el balance de energía aplicaremos el Primer Principio y su ecuación para sistemas abiertos. Esta tiene la siguiente forma:

\begin{equation}
\dot{Q} - \dot{W} + \sum_e \dot{m}_e \left(h + e_c + e_p \right)_e = \sum_s \dot{m}_s\left(h + e_c + e_p\right)_s 
\end{equation}

Vamos a desmenuzar la ecuación: primero diremos que, obviamente, todos los términos representan diferentes tipos de energía.

Los términos $\dot{Q}$ y $\dot{W}$ representan calor y trabajo respectivamente: son las dos maneras en que un sistema intercambia energía con el entorno. Discutiremos esto más adelante. El punto sobre la letra indica que es calor por unidad de tiempo.

\subsection{Ejemplo: Balance de materia y energía en la Central Térmica Güemes}
La Central Térmica Güemes, ubicada en Salta, Argentina, opera con un ciclo Rankine. Supongamos que analizamos el flujo de agua en la caldera:

\begin{itemize}
\item Flujo de entrada de agua líquida: $\dot{m}_e = 600\,\text{kg/s}$
\item Flujo de salida de vapor de agua: $\dot{m}_s = 600\,\text{kg/s}$
\item Entalpía del agua líquida: $h_e = 800\,\text{kJ/kg}$
\item Entalpía del vapor de salida: $h_s = 2800\,\text{kJ/kg}$
\end{itemize}

Dado que no hay acumulación de materia en estado estacionario:

\begin{equation}
600\,\text{kg/s} = 600\,\text{kg/s}
\end{equation}

Para el balance de energía:

\begin{equation}
\dot{Q} = \dot{m} (h_s - h_e) = 600 (2800 - 800) = 1200000\,\text{kW}
\end{equation}

Esto confirma que la energía térmica suministrada permite la transformación de agua en vapor dentro de la caldera.







