\documentclass[12pt,a4paper]{article}

% --- Paquetes Recomendados ---
\usepackage[utf8]{inputenc}
\usepackage[spanish]{babel}
\usepackage{amsmath}
\usepackage{amsfonts}
\usepackage{amssymb}
\usepackage{graphicx}
\usepackage{hyperref}
\usepackage{geometry}
\usepackage{enumitem}
\usepackage{titlesec}
\usepackage{booktabs} % Para tablas profesionales

% --- Configuración de Página ---
\geometry{margin=1in}
\titleformat{\section}{\large\bfseries}{\thesection}{1em}{}[\titlerule]

% --- Información del Documento ---
\title{
    \textbf{Proyecto MAYER} \\
    \large Plataforma Integrada para la Enseñanza de Termodinámica Técnica 
}
\author{Dr. Pablo S. Gauna}
\date{}

\begin{document}

\maketitle

\section{Identidad del Proyecto}
El \textbf{Proyecto MAYER} bla bla

\section{Arquitectura del Ecosistema Multicanal}
El proyecto se estructura mediante la integración de cuatro pilares estratégicos que funcionan de forma sinérgica:

\begin{itemize}[leftmargin=1.5em]
    \item \textbf{Tomo I: Plantas de Generación de Potencia (Editorial UNGS):} Texto de referencia que sustituye ejercicios genéricos por casos de estudio del parque generador nacional, con foco en el análisis exergético.
    \item \textbf{Observatorio Energético Regional (Sección A):} Base de datos técnica y operativa de centrales térmicas (Central Güemes, Atucha II, Costanera, etc.), que actúa como el laboratorio real del libro.
    \item \textbf{Software de Simulación (App Python):} Motor de cálculo de estados termodinámicos mediante \textit{Cubic Splines}, permitiendo una visualización dinámica y precisa de ciclos reales en diagramas $T-s$.
    \item \textbf{Canal de Ingeniería Visual (YouTube):} Plataforma de narrativa técnica destinada a la contextualización de activos energéticos y la explicación de fenómenos complejos en planta.
\end{itemize}

\section{Plan de Gestión y Designación de Tareas}

\begin{table}[h]
\centering
\caption{Hoja de Ruta Operativa - Proyecto MAYER}
\vspace{0.5em}
\begin{tabular}{@{}lll@{}}
\toprule
\textbf{Área} & \textbf{Tarea Crítica} & \textbf{Entregable} \\ \midrule
\textbf{Investigación} & Digitalización de Tablas de Vapor & Dataset Curado (CSV/JSON) \\
\textbf{Desarrollo} & Programación del Motor Termodinámico & Web-App Interactiva \\
\textbf{Editorial} & Redacción de Capítulos y Observatorio & Manuscrito Tomo I \\
\textbf{Contenido} & Guionado y Producción Audiovisual & Serie ``Termodinámica Real'' \\ \bottomrule
\end{tabular}
\end{table}
\clearpage

\section{Estrategia de Financiamiento y Publicación}
\begin{itemize}
    \item \textbf{Eje Científico:} Publicaciones sobre eficiencia del SADI en revistas de energía y aplicación a fondos PICT / Proyectos UTN.
    \item \textbf{Eje Pedagógico:} Artículos sobre el impacto de la enseñanza inversa y herramientas Open Source en congresos de ingeniería (CAEDI/CONFEDI).
\end{itemize}

\section{Hitos Próximos}
\begin{enumerate}
    \item Consolidación de la Sección A (Datos técnicos de la Central Güemes).
    \item Lanzamiento del prototipo de la App con motor de interpolación Spline.
    \item Presentación del plan de obra ante el consejo editorial de la UNGS.
\end{enumerate}

\end{document}